\chapter{Spannungsfall}
\label{cha:Kapitel5}

Das Thema Spannungsfall ist besonders wichtig bei einer Leitungsberechnung. Oft genug erleben wir die Auswirkungen einer schlecht dimensionierten Leitung. Jeder von uns kennt die folgende Situation. Wir kommen in einer �lteren Wohnung an und setzen uns gem�tlich auf das Sofa. Danach machen den Deckenfluter an, um den Raum mit einem angenehmen Licht zu fluten. Schalten wir nun den Fernseher an sehen wir sofort die Auswirkungen jener alten schlecht dimensionierten Leitung. Das Licht wird f�r einen kurzen Moment \textsc{merklich dunkler}. Diese Situation ist besonders �rgerlich, wenn Verbraucher mit hohen Leistungen (Herd, Waschmaschine, Trockner, Geschirrsp�lmaschine) �fters eingeschaltet werden. Das Licht flackert und der gem�tliche Abend ist dahin!

Aus diesem Grund ist das Thema Spannungsfall mit einer der wichtigsten Punkte. Hier geht es um die Kundenzufriedenheit nach der Verlegung unserer Leitung.

\section{Grundlagen zum Spannungsfall}
\label{cha:Grundlagen zum Spannungsfall}



\section{Grenzwerte und gesetzliche Bestimmungen}
\label{cha:Grenzwerte und gesetzliche Bestimmungen}

\subsection{Grenzwerte nach DIN 18015 Teil 1}

{\color{red}\textbf{Der max. erlaubte Spannungsfall betr�gt 3 \% ab dem Z�hler bis zur Anschlussstelle der Verbraucher.}}

In der Praxis hat es sich leider durchgesetzt, den Spannungsfall zu vernachl�ssigen. Fakt ist, es sind max. 3 \% erlaubt. Wird diese Grenze �berschritten, so kann dies dazu f�hren, dass im Fehlerfall kein Versicherungsschutz mehr besteht und der Elektriker auf den Kosten sitzen bleibt. 

Beispiel des Spannungsfalls in Verbindung mit einer UV:

Hauptverteilung mit Z�hler $ \longrightarrow $ Unterverteilung $ \longrightarrow $ Anschlussstelle des Verbrauchers \\

Hauptverteilung mit Z�hler $ \longrightarrow $ Unterverteilung = 1 \% Spannungsfall\\

Unterverteilung $ \longrightarrow $ Anschlussstelle des Verbraucher = 2 \% Spannungsfall\\


\subsection{TAB}

Die TAB aus Niedersachsen schreibt folgende Grenzwerte auf der Hauptleitung vor: (Hauptleitung = Hausanschlusskasten $ \longrightarrow $ Z�hler).

\begin{table}[h]

	\caption{Spannungsfall nach TAB}

	\label{fig:Spannungsfall nach TAB}

	\centering

	\begin{tabular}{|c||c||c|}

		\hline Von & Bis & Spannungsfall \\ 

		\hline\hline  & 100 kVA & 0,5 \% \\ 

		\hline 100 kVA & 250 kVA & 1  \% \\ 

		\hline 250 kVA & 400 kVA & 1,25 \% \\ 

		\hline 400 kVA &  & 1,5 \% \\ 

		\hline 

	\end{tabular}

\end{table}


\subsection{VDE 0100 Teil 520}


{\color{red}\textbf{Die VDE schreibt einen Grenzwert von 5 \% f�r andere elektrische Verbraucher und 3\% f�r Beleuchtung vor.}}


Die 5 \% sollten auch �ber die Verl�ngerungsleitungen ausgedehnt werden, die an den Steckdosen angeschlossen sind (Empfehlung VDE).

Das hei�t: 5 \% vom Hausanschlusskasten bis zum Endverbraucher inklusive seiner Zuleitung.


\section{Formelsammlung}
\label{cha:Formelsammlung}

Es gibt verschiedenen M�glichkeiten, die Leitung, nach dem Spannungsfall, zu verifizieren. Man unterscheidet weiterhin, bei den Formeln, zwischen Wechselstrom und Drehstromverbrauchern. Weiterhin ist der Strom wichtig, welcher in der Berechnung zu verwenden ist. Wir orientieren uns an Kapitel \ref{cha:Welcher Strom wird in der Berechnung verwendet?} und benutzten die dort beschriebenen Str�me.


\subsection{Formel f�r Wechselstromverbraucher}


Berechnung des Spannungsfalles in \%:

\begin{equation}
\Delta u = \frac{200 \cdot I \cdot L \cdot \cos\varphi}{K \cdot A \cdot U}
\label{eq:Wechselstrom Spannungsfall Prozent}
\end{equation}

\begin{equation}
A = \frac{200 \cdot I \cdot L \cdot \cos\varphi}{K \cdot \Delta u \cdot U}
\label{eq:Wechselstrom Querschnitt Prozent}
\end{equation}

\begin{list}{}{}
	\item $ L $ = L�nge der Leitung
	\item $ I $ = Strom nach Kapitel \ref{cha:Welcher Strom wird in der Berechnung verwendet?}
	\item $ \cos\varphi  $ = Leistungsfaktor des Ger�tes, welches an die Leitung angeschlossen wird
	\item $ A $ = Querschnitt der zu berechnenden Leitung
	\item $ K $ = Elektrische Leitf�higkeit der Leitung, bei Kupfer $ K = 56 \cdot \frac{\text{m}}{\text{mm}^{2} \cdot\,\Omega} $
	\item $ U $ = Spannung. Hier 230 V!
	\item $ \Delta u $ = Spannungsfall in \%
\end{list}

\subsection{Formeln f�r Drehstromverbraucher}


Berechnung des Spannungsfalles in \%:

\begin{equation}
\Delta u = \frac{\sqrt{3} \cdot 100 \cdot I \cdot L \cdot \cos\varphi}{K \cdot A \cdot U}
\label{eq:Drehstrom Spannungsfall Prozent}
\end{equation}

\begin{equation}
A = \frac{\sqrt{3} \cdot 100 \cdot I \cdot L \cdot \cos\varphi}{K \cdot \Delta u \cdot U}
\label{eq:Drehstrom Querschnitt Prozent}
\end{equation}

\begin{list}{}{}
	\item $ L $ = L�nge der Leitung
	\item $ I $ = Strom nach Kapitel \ref{cha:Welcher Strom wird in der Berechnung verwendet?}
	\item $ \cos\varphi  $ = Leistungsfaktor des Ger�tes, welches an die Leitung angeschlossen wird
	\item $ A $ = Querschnitt der zu berechnenden Leitung
	\item $ K $ = Elektrische Leitf�higkeit der Leitung, bei Kupfer $ K = 56 \cdot \frac{\text{m}}{\text{mm}^{2} \cdot\,\Omega} $
	\item $ U $ = Spannung. Hier 400 V!
	\item $ \Delta u $ = Spannungsfall in \%
\end{list}


\subsection{Beispiele zur Berechnung des Spannungsfalls}

\subsubsection{Beispiel f�r eine Herdzuleitung}


Die L�nge der Herdzuleitung betr�gt 20 m und sie ist mit einem Leitungsschutzschalter Typ B16 A abgesichert. Es wurden durch andere Berechnungen bereits ein Querschnitt von $ A = 2,5 \,\text{mm}^{2} $ ermittelt. Wir ermitteln nun nach Formel \ref{eq:Drehstrom Spannungsfall Prozent} den auftretenden Spannungsfall:


\[ \Delta u = \frac{\sqrt{3} \cdot 100 \cdot I \cdot L \cdot \cos\varphi}{K \cdot A \cdot U} \]

\[ = \frac{\sqrt{3} \cdot 100 \cdot 16\,\text{A} \cdot 20\,\text{m} \cdot 1}{56 \cdot \frac{\text{m}}{\text{mm}^{2} \cdot\,\Omega} \cdot 2,5 \,\text{mm}^{2} \cdot 400\,\text{V}} = 1\,\text{\%} \]


Der Spannungsfall h�lt alle Grenzwerte nach Kapitel \ref{cha:Grenzwerte und gesetzliche Bestimmungen} ein. Der Querschnitt $ A = 2,5\,\text{mm}^{2} $ wurde verifiziert.


\subsubsection{Beispiel f�r eine Motorzuleitung}


Die L�nge der Motorzuleitung betr�gt 18 m. Der Motor wird direkt und ohne Steckvorrichtungen angeschlossen. Die Schmelzsicherung ist vom Typ 10 A gL-gG und der Betriebsstrom betr�gt I = 4,6 A und der Leistungsfaktor $ \cos\varphi = 0,82 $. Als Zuleitung wurde ein Querschnitt von $ A = 1,5 \,\text{mm}^{2} $ ermittelt. Nach Formel \ref{eq:Drehstrom Spannungsfall Prozent} gilt:


\[ \Delta u = \frac{\sqrt{3} \cdot 100 \cdot I \cdot L \cdot \cos\varphi}{K \cdot A \cdot U} \]

\[ = \frac{\sqrt{3} \cdot 100 \cdot 4,6\,\text{A} \cdot 18\,\text{m}  \cdot 0,82}{56 \cdot \frac{\text{m}}{\text{mm}^{2} \cdot\,\Omega} \cdot 1,5 \,\text{mm}^{2} \cdot 400\,\text{V}} = 0,35\,\text{\%} \]


Der Spannungsfall h�lt alle Grenzwerte nach Kapitel \ref{cha:Grenzwerte und gesetzliche Bestimmungen} ein. Der Querschnitt $ A = 1,5\,\text{mm}^{2} $ wurde verifiziert.


\section{Spannungsfall f�r dein Handy}

Hier kannst du dir die aktuelle App f�r dein iPhone runder laden:

\begin{figure}[htbp]
	\centering
	\quad
	\qrcode[hyperlink, height=1in]{https://itunes.apple.com/de/app/spannungsfall/id574470525?mt=8}
	\qquad
\end{figure}