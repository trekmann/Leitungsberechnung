\part{Anhang}
\label{cha:Anhang 1}

\chapter{Versionen}
\label{cha:Versionen}

\begin{itemize}
\item \textbf{Version 1.0: } Erste Version des Dokuments. Dieses wurde mit MS-Word erstellt.
\item \textbf{Version 1.1 - 1.4: } Diverse Fehlerbehebungen.
\item \textbf{Version 1.5: } Das Dokument wurde von MS Office 2003 f�r Windows auf MS Office 2008 f�r Mac ungeschrieben. Dabei ist das Kapitel mit den Oberwellen hinzugekommen. Einige Fehler wurden dabei behoben.
\item \textbf{Version 2.0: }

	\begin{itemize}
	\item Neuschaffung des Dokumentes Leitungsberechnung mit \LaTeX. Grundlage war die Version 1.5 der Leitungsberechnung.pdf.
	\item Hinzugef�gt wurde die Tabelle \ref{fig:Strombelastbarkeit Erdverlegung} zum Thema Belastbarkeit von Kabel und Leitungen in der Erde.
	\item Diverse Fehler der Version 1.5 wurden beseitigt.
	\item Das Kapitel 7 Oberwellen wurde entfernt und in eine eigene Datei ausgelagert.
	\end{itemize}

\item \textbf{Version 2.0.1 - 2.0.5: } Diverse Fehlerbehebungen.
\item \textbf{Version 2.1: }

	\begin{itemize}
	\item Diverse Fehlerbehebungen.
	\item Kapitel \ref{cha:Versionen} erg�nzt, damit die Entwicklung des Programms nachverfolgt werden kann.
	\item Kapitel \ref{cha:Schnellauswahl einer Leitung in der Praxis} wurde erg�nzt. Hiermit soll es m�glich sein einfach Leitungen zu dimensionieren ohne Berechnung! Die g�ngigsten Parameter werden in Tabellen festgehalten und der richtige Querschnitt muss nur noch in den Tabellen ermittelt werden!
	\end{itemize}

\item \textbf{Version 2.2: }

	\begin{itemize}
	\item Neues Logo eingef�hrt f�r dieLeitungsberechnung! Mein besonderer Dank geht an die Designerin Julia Eggeling!
	\item �berarbeitung des Deckblatts
	\item �berarbeitung der Kopfzeile
	\item Kapitel 2 �berarbeitet
	\item Kleine Fehler beseitigt
	\end{itemize}
	
\item \textbf{Version 2.2.1: }Das Deckblatt wurde um den Link zur Webseite erg�nzt!

\item \textbf{Version 3.0: } 

	\begin{itemize}
	\item Die Schnellauswahl wurde an den Anfang gestellt
	\item Die Tabellen f�r Belastbarkeit, H�ufung und Temperatur wurden in die passenden Kapitel verschoben
	\item Das Kapitel \ref{cha:Grundlagen Leitungsschutzschalter} wurde hinzugef�gt, zum Thema Leitungsschutzschalter
	\item Das Kapitel \ref{cha:Grundlagen Schmelzsicherungen} wurde hinzugef�gt, zum Thema Schmelzsicherungen
	\item Das Kapitel \ref{cha:Selektivit�t bei der Auswahl von Sicherungen} wurde hinzugef�gt, zum Thema Selektivit�t bei der Auswahl von Sicherungen
	\end{itemize}
	
\item \textbf{Version 3.1: } 

	\begin{itemize}
	\item Einheiten werden einheitlich mit Leerzeichen und nicht kursiv dargestellt
	\item Fehlende Verweise in den Bildern von Hager hinzugef�gt
	\item Fehler in der Normung der Achsen von den Diagrammen beseitigt
	\item Diverse Verbesserungen am Layout
	\item Hinweis hinzugef�gt, dass es sich um Leitungen aus Kupfer handelt
	\item Diverse kleine Fehlerkorrekturen
	\end{itemize}
	
\item \textbf{Version 3.2: } 

Diverse Fehlerkorrekturen

\item \textbf{Version 4.0: } 

\begin{itemize}
\item Neugestaltung von Kapitel \ref{cha:Kapitel5}
\item Neues Icon auf der ersten Seite
\item Diverse Fehlerkorrekturen
\end{itemize}

\end{itemize}